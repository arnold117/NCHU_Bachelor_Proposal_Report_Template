%%
% NCHU Bachelor Proposal Report Template
%
% 南昌航空大学毕业设计开题报告 —— 使用 XeLaTeX 编译
%
% Copyright 2023 Arnold Chow
%
% The Current Maintainer of this work is Arnold Chow.
%
% Compile with: xelatex -> biber -> xelatex -> xelatex

% 章节支持、单面打印:ctexbook
\documentclass[UTF8,AutoFakeBold,AutoFakeSlant,zihao=-4,oneside,openany]{ctexbook}
\usepackage[a4paper,left=3.5cm,right=2.5cm,top=2.5cm,bottom=2.5cm]{geometry}
% 目前 29mm 最接近 Word 排版
\usepackage{xeCJK}
\usepackage{titletoc}
\usepackage{fontspec}
\usepackage{setspace}
\usepackage{graphicx}
\usepackage{fancyhdr}
\usepackage{pdfpages}
\usepackage{setspace}
\usepackage{booktabs}
\usepackage{multirow}
\usepackage{caption}
\usepackage{tikz}
\usepackage{etoolbox}
\usepackage{hyperref}
\usepackage{xcolor}
\usepackage{caption}
\usepackage{array}
\usepackage{amsmath}
\usepackage{amssymb}
\usepackage{pdfpages}
\usepackage{float}
\usepackage[section]{placeins}
\usepackage{enumerate}
\usepackage{ulem}

% 设置参考文献编译后端为 biber,引用格式为 GB/T7714-2015 格式
% 参考文献使用宏包见 https://github.com/hushidong/biblatex-gb7714-2015
\usepackage[
  backend=biber,
  style=gb7714-2015,
  gbalign=gb7714-2015,
  gbnamefmt=lowercase,
  gbpub=false,
  doi=false,
  url=false,
  eprint=false,
  isbn=false,
]{biblatex}

% 参考文献引用文件位于 misc/ref.bib
\addbibresource{misc/ref.bib}

% 西文字体默认为 Times New Roman
\setromanfont{Times New Roman}
% 论文题目字体为华文细黑
\setCJKfamilyfont{xihei}[Path=fonts/]{STXIHEI.TTF} % 若希望使用本机字体,也可以用 {STXihei} 来调用
\newcommand{\xihei}{\CJKfamily{xihei}}

% 论文中文题目
\newcommand{\thesisTitle}{毕业设计(论文)题目}
% 论文英文题目(可选)
\newcommand{\thesisTitleEN}{The Subject of Undergraduate Graduation Project (Thesis)}

% 在这里填写你的相关信息
\newcommand{\deptName}{测试与光电工程学院}
\newcommand{\majorName}{生物医学工程}
\newcommand{\yourName}{周XX}
\newcommand{\yourStudentID}{190841XX}
\newcommand{\mentorName}{余XX}
% 如果你的毕设为校外毕设,请将下面这一行语句解除注释(删除第一个百分号字符)并在第二组花括号中填写你的校外毕设导师名字
% \newcommand{\externalMentorName}{左偏树}

% 主题页面格式:NCHUThesis
\fancypagestyle{NCHUThesis}{
  % 设置空页眉
  \fancyhead{}
  % 删除页眉横线
  \renewcommand{\headrulewidth}{0pt}
  % 页码高度(不完美,比规定稍微靠下 2mm)
  \setlength{\footskip}{14pt}
  % 定义页码
  \fancyfoot[C]{\songti\zihao{-5} \thepage}
}

% 设置章节格式,适用于说明页
% 一级标题:宋体,小三号,加粗;间距:段前 0.5 行,段后 1 行;
\ctexset{chapter={
    number = {\arabic{chapter}},
    format = {\songti \bfseries \centering \zihao{-3}},
    aftername = \hspace{9bp},
    pagestyle = NCHUThesis,
    beforeskip = 8bp,
    afterskip = 32bp,
    fixskip = true,
  }
}

% 二级标题:黑体,小三号,加粗,汉字编号;间距:段前 0.5 行,段后 0 行;
\ctexset{section={
    number = {\chinese{section}},
    format = {\xihei \raggedright \bfseries \zihao{4}},
    aftername = \xihei{、},
    beforeskip = 20bp plus 1ex minus .2ex,
    afterskip = 18bp plus .2ex,
    fixskip = true,
  }
}

% 设置目录样式
% 添加 PDF 链接
\addtocontents{toc}{\protect\hypersetup{hidelinks}}

% 修改超链接、引用的颜色
\hypersetup{
  colorlinks=true,
  linkcolor=black,
  anchorcolor=black,
  citecolor=black
}

% 前置页面(说明)
\renewcommand{\frontmatter}{
  \pagenumbering{Roman}
  \pagestyle{NCHUThesis}
}

% 正文页面
\renewcommand{\mainmatter}{
  \pagenumbering{arabic}
  \pagestyle{NCHUThesis}
}

% % 设置 caption 与 figure 之间的距离
% \setlength{\abovecaptionskip}{11pt}
% \setlength{\belowcaptionskip}{9pt}

% % 设置图片的 caption 格式
% \renewcommand{\thefigure}{\thechapter-\arabic{figure}}
% \captionsetup[figure]{font=small,labelsep=space}

% % 设置表格的 caption 格式和 caption 与 table 之间的垂直距离
% \renewcommand{\thetable}{\thechapter-\arabic{table}}
% \captionsetup[table]{font=small,labelsep=space,skip=2pt}

%%% ---- 图表标题设置 ----- %%%
\RequirePackage[labelsep=quad]{caption}     % 序号之后空一格写标题
% 设置表格标题字体为黑体, 设置图标题字体为宋体
\DeclareCaptionFont{heiti}{\heiti}
\captionsetup[table]{textfont=heiti}
\renewcommand\figurename{\songti\zihao{-4} 图}  
\renewcommand\tablename{\heiti\zihao{-4} 表} 

% 使用tabularx创建占满宽度的表格
\RequirePackage{tabularx, makecell}
\newcolumntype{L}{X}
\newcolumntype{C}{>{\centering \arraybackslash}X}
\newcolumntype{R}{>{\raggedleft \arraybackslash}X}

\RequirePackage{longtable}  % 做长表格的包
\RequirePackage{booktabs}   % 做三线表的包

% 列表样式
\RequirePackage{enumerate, enumitem}
\setlist{noitemsep}

% 调整底层 TeX 排版引擎参数以保证所有段落能够很好地以两端对齐的方式呈现
\tolerance=1
\emergencystretch=\maxdimen
\hyphenpenalty=10000
\hbadness=10000

% 设置数学公式编号格式
\renewcommand{\theequation}{\arabic{chapter}.\arabic{equation}}

\newcommand{\unnumchapter}[1]{
  \chapter*{\vskip 10bp\textmd{#1} \vskip -6bp}
  \addcontentsline{toc}{chapter}{#1}
  \stepcounter{chapter}
}

% 公式引用使用中文括号
\renewcommand{\eqref}[1]{\textup{{\normalfont(\ref{#1})\normalfont}}}

%%% ---- 引入宏包 ----- %%%
\RequirePackage{amsmath, amssymb}
\RequirePackage[amsmath,thmmarks]{ntheorem}  % 定理
\RequirePackage{graphicx, subcaption}
\RequirePackage{listings}                    % 代码段
% \RequirePackage{minted}                    % 代码高亮(需要 python 安装 pygments 库)
\RequirePackage[ruled,vlined]{algorithm2e}
\RequirePackage{algorithmic}    % 算法代码
\RequirePackage{tikz, pgfplots}              % 绘图
\RequirePackage{fontspec, color, url, array}

\RequirePackage{txfonts}                     % Times 风格(数学)字体

%%% ---- 定义字体 ----- %%%
\renewcommand{\normalsize}{\zihao{-4}}         % 正常字号
% 设置英文字体为 Times New Roman
\setmainfont[Ligatures=Rare]{Times New Roman}
\setsansfont[Ligatures=Rare]{Times New Roman}
\setmonofont[Ligatures=Rare]{Times New Roman}

% 算法两字用中文显示
\renewcommand{\algorithmcfname}{算法}

\lstdefinestyle{code}{
	backgroundcolor=\color{gray!10},
	commentstyle=\color{green!50!black},
	keywordstyle=\color{blue},
	stringstyle=\color{magenta},
	basicstyle=\linespread{1}\footnotesize\ttfamily,
	numberstyle=\tiny,
	breakatwhitespace=false,
	breaklines=true,
	captionpos=t,
	frame=single,
	keepspaces=true,
	language=java,
	numbers=none,
	numbersep=5pt,
	showspaces=false,
	showstringspaces=false,
	showtabs=false,
	tabsize=2,
	aboveskip=1em,
	belowskip=1em,
	belowcaptionskip=12pt
}

% 修改脚注
\makeatletter%
\long\def\@makefnmark{%
\hbox {{\normalfont \textsuperscript{\circled{\@thefnmark}}}}}%
\makeatother
\makeatletter%
\long\def\@makefntext#1{%
  \parindent 1em\noindent \hb@xt@ 1.8em{\hss \circled{\@thefnmark}}#1}%
\makeatother
\skip\footins=10mm plus 1mm
\footnotesep=6pt
\renewcommand{\footnotesize}{\songti\zihao{5}}
\renewcommand\footnoterule{\vspace*{-3pt}\hrule width 0.3\columnwidth height 1pt \vspace*{2.6pt}}

\newcommand*\circled[1]{\tikz[baseline=(char.base)]{%
\node[shape=circle,draw,inner sep=0.5pt] (char) {#1};}} % 圆圈数字①

%%% ---- 数学定理样式 ----- %%%
\theoremstyle{plain}
\theoremheaderfont{\heiti}
\theorembodyfont{\songti} \theoremindent0em
\theorempreskip{0pt}
\theorempostskip{0pt}
\theoremnumbering{arabic}
%\theoremsymbol{} %定理结束时自动添加的标志
\newtheorem{theorem}{\hspace{2em}定理}[section]
\newtheorem{definition}{\hspace{2em}定义}[section]
\newtheorem{lemma}{\hspace{2em}引理}[section]
\newtheorem{corollary}{\hspace{2em}推论}[section]
\newtheorem{proposition}{\hspace{2em}性质}[section]
\newtheorem{example}{\hspace{2em}例}[section]
\newtheorem{remark}{\hspace{2em}注}[section]

\theoremstyle{nonumberplain}
\theoremheaderfont{\heiti}
\theorembodyfont{\normalfont \rm \songti}
\theoremindent0em \theoremseparator{\hspace{1em}}
\theoremsymbol{$\square$}
\newtheorem{proof}{\hspace{2em}证明}

% 文档开始
\begin{document}

% 标题页面:如无特殊需要,本部分无需改动
\input{misc/0_cover.tex}

% 前置页面定义
\frontmatter
\newpage
% 说明页面:如无特殊需要,本部分无需改动
%%
% NCHU Bachelor Proposal Report Template
%
% 南昌航空大学毕业设计开题报告 (说明页) —— 使用 XeLaTeX 编译
%
% Copyright 2023 Arnold Chow
%
% The Current Maintainer of this work is Arnold Chow.
%
% 说明

\unnumchapter{说~~~~明}

开题报告应结合自己课题而作,一般包括:课题依据及课题的意义、
国内外研究概况及发展趋势(含文献综述)、研究内容及实验方案、目标、
主要特色及工作进度、参考文献等内容。
\textbf{以下填写内容各专业可根据具体情况适当修改,但同一专业应保持一致。}
\newpage

% 正文开始
\mainmatter
% 正文 22 磅的行距
\setlength{\parskip}{0em}
\renewcommand{\baselinestretch}{1.53}
% 修复脚注出现跨页的问题
\interfootnotelinepenalty=10000

% 按二级标题添加
% 选题的依据及意义
%%
% NCHU Bachelor Proposal Report Template
%
% 南昌航空大学毕业设计开题报告(选题的依据与意义)—— 使用 XeLaTeX 编译
%
% Copyright 2023 Arnold Chow
%
% The Current Maintainer of this work is Arnold Chow.
%
% Compile with: xelatex -> biber -> xelatex -> xelatex

\section{选题的依据及意义}

\textbf{前言: \LaTeX 的简单使用:}

在这里插入一个参考文献,仅作参考\cite{yuFeiJiZongTiDuoXueKeSheJiYouHuaDeXianZhuangYuFaZhanFangXiang2008}。
可以通过空一行(两次回车)实现段落换行,仅仅是回车并不会产生新的段落。

正文……\cite{simonyanVeryDeepConvolutional2015}

常使用的其他字体格式:

{\songti \bfseries 宋体加粗} {\textbf{English}}

{\songti \itshape 宋体斜体} {\textit{English}}

{\songti \bfseries \itshape 宋体粗斜体} {\textbf{\textit{English}}}

\textbf{注意!}

本模板必须使用 XeLaTeX + BibTeX 编译,否则会直接报错。
本模板支持多个平台,结合 Sublime Text/VSCode/Overleaf 都可以使用。
本模板支持中英文混排,但是中文和英文之间必须有空格,否则会报错。

\vspace{10mm}

该部分主要需要写:
\begin{enumerate}[label=\arabic*)]
    \item 研究课题的名称是什么?为什么要进行该课题的研究?
    \item 研究的目标是什么?为实现总目标,需要实现哪些子目标?
    \item 研究的预期成果是什么?该成果具有什么科学价值?
\end{enumerate}

% 国内外研究概况和发展趋势
%%
% NCHU Bachelor Proposal Report Template
%
% 南昌航空大学毕业设计开题报告(国内外研究概况和发展趋势)—— 使用 XeLaTeX 编译
%
% Copyright 2023 Arnold Chow
%
% The Current Maintainer of this work is Arnold Chow.
%
% Compile with: xelatex -> biber -> xelatex -> xelatex

\section{国内外研究概况及发展趋势}

(含文献综述)

该部分主要需要写:
\begin{enumerate}[label=\arabic*)]
    \item 要实现研究目标,要使用哪些方法、系统、工具和技术?国内外研究现状、发展动态如何?
    \item 是否还有其它方法、系统、工具和技术?分析、比较它们各自有哪些优缺点?
\end{enumerate}
这就需要大家多阅读文献。关于文献还需要注意:
\begin{enumerate}[label=\arabic*)]
    \item 是否多为近3-5年的参考文献,且来自本领域的主流期刊?
    \item 参考文献的引用格式是否规范?(\sout{都使用这套模板了,那肯定没问题的啦})
\end{enumerate}

% 研究内容及实验方案
%%
% NCHU Bachelor Proposal Report Template
%
% 南昌航空大学毕业设计开题报告(研究内容及实验方案)—— 使用 XeLaTeX 编译
%
% Copyright 2023 Arnold Chow
%
% The Current Maintainer of this work is Arnold Chow.
%
% Compile with: xelatex -> biber -> xelatex -> xelatex

\section{研究内容及实验方案}
要求尽量详尽,应有500字以上。

研究内容:
\begin{enumerate}[label=\arabic*)]
    \item 要实现研究的子目标,分别要对哪些方面进行深入研究?这些研究对实现总目标分别起什么样的作用?
    \item 研究的特色和新意是什么?拓展理论、革新技术、完善方法、解决争议、验证假说等?
\end{enumerate}

研究方案:
\begin{enumerate}[label=\arabic*)]
    \item 实现研究目标的总体方案是什么?要完成总体方案,必须完成哪几项工作,分别起到什么作用?
    \item 各项工作中需要解决的关键问题是什么?用什么关键技术来解决这些关键问题?
    \item 用系统结构图/流程图/逻辑图等方式对技术路线加以说明。
\end{enumerate}

同时,还需要对困难进行评估:
\begin{enumerate}[label=\arabic*)]
    \item 在开展研究时,技术上是否存在障碍,有没有办法克服?
    \item 如果研究方案出现行不通的情况,是否有备用方案?
\end{enumerate}
% 目标、主要特色及工作进度
%%
% NCHU Bachelor Proposal Report Template
%
% 南昌航空大学毕业设计开题报告(目标、主要特色及工作进度)—— 使用 XeLaTeX 编译
%
% Copyright 2023 Arnold Chow
%
% The Current Maintainer of this work is Arnold Chow.
%
% Compile with: xelatex -> biber -> xelatex -> xelatex

\section{目标、主要特色及工作进度}
清晰描述毕业设计的目标,如做出了个具有什么样功能的实物,有和其他研究怎么不一样的特色。

关于工作进度:
\begin{enumerate}[label=\arabic*)]
    \item 以往学习、研究、实践等工作中与本课题直接或间接相关的研究基础?
    \item 有哪些主客观条件是对本研究工作有利的?
    \item 对于尚不具备的条件采取什么措施,以确保研究工作的顺利完成?
\end{enumerate}


% 参考文献
\input{misc/2_reference.tex}
% 导师意见
%%
% NCHU Bachelor Proposal Report Template
%
% 南昌航空大学毕业设计开题报告(指导老师意见)—— 使用 XeLaTeX 编译
%
% Copyright 2023 Arnold Chow
%
% The Current Maintainer of this work is Arnold Chow.
%
% Compile with: xelatex -> biber -> xelatex -> xelatex

\section{指导教师意见}

\textcolor{blue}{指导教师提出具体意见,并表明是否同意毕业设计(论文)开题。}
\textcolor{blue}{此部分需手写并签名,请在阅读后删除该段文字。}


\end{document}
